\documentclass{tufte-handout} % - https://tufte-latex.github.io/tufte-latex/

\title{Problem Set 1\thanks{IAP 2020 18.S097:\\Programming with Categories}}

\author{Alexey Filippov\thanks{Kamchatka Ltd}}

%\date{12 January 2020} % without \date command, current date is supplied

%\geometry{showframe} % display margins for debugging page layout


\usepackage{graphicx} % allow embedded images
  \setkeys{Gin}{width=\linewidth,totalheight=\textheight,keepaspectratio}
  \graphicspath{{graphics/}} % set of paths to search for images
\usepackage{amsmath}  % extended mathematics

\usepackage{stmaryrd} % hollow semicolon - https://tex.stackexchange.com/questions/125012/what-is-the-latex-code-for-an-open-semicolon/125032

% proper empty set sign - https://tex.stackexchange.com/questions/22798/nice-looking-empty-set
\usepackage{amssymb}
\let\oldemptyset\emptyset
\let\emptyset\varnothing

% for lambda characters - https://tex.stackexchange.com/questions/34604/entering-unicode-characters-in-latex
\usepackage[utf8]{inputenc} 
\DeclareUnicodeCharacter{03BB}{\lambda}

\usepackage{booktabs} % book-quality tables
\usepackage{units}    % non-stacked fractions and better unit spacing
\usepackage{multicol} % multiple column layout facilities
\usepackage{lipsum}   % filler text
\usepackage{fancyvrb} % extended verbatim environments
  \fvset{fontsize=\normalsize}% default font size for fancy-verbatim environments

\usepackage{minted}

% Standardize command font styles and environments
\newcommand{\doccmd}[1]{\texttt{\textbackslash#1}}% command name -- adds backslash automatically
\newcommand{\docopt}[1]{\ensuremath{\langle}\textrm{\textit{#1}}\ensuremath{\rangle}}% optional command argument
\newcommand{\docarg}[1]{\textrm{\textit{#1}}}% (required) command argument
\newcommand{\docenv}[1]{\textsf{#1}}% environment name
\newcommand{\docpkg}[1]{\texttt{#1}}% package name
\newcommand{\doccls}[1]{\texttt{#1}}% document class name
\newcommand{\docclsopt}[1]{\texttt{#1}}% document class option name
\newenvironment{docspec}{\begin{quote}\noindent}{\end{quote}}% command specification environment

\newcommand{\ito}{It\^o}
\newcommand{\ttoT}{\stackrel{t \rightarrow T}{\sim}}
\newcommand{\expect}[1]{\mathop{\mathbb{E}\left[ #1\right]}}
\newcommand{\pd}[2]{\frac{\partial #1}{\partial #2}}
\newcommand{\pdd}[2]{\frac{\partial^2 #1}{\partial {#2}^2}}
\newcommand{\od}[2]{\frac{d #1}{d #2}}
\newcommand{\odd}[2]{\frac{d^2 #1}{d {#2}^2}}
\newcommand{\half}{\frac{1}{2}}

\DeclareMathOperator\erfc{erfc}

\begin{document}

\maketitle% this prints the handout title, author, and date
\begin{abstract}
This document is the report part of the homework.
The required results and diagrams are provided; the source
code is only included where it was necessary to illustrate the approach.

Attached, please find the full Haskell source code for all the tasks.
\end{abstract}


%% ps1_q1.tex
\section{ Functions in mathematics and Haskell. } 
We're tasked with implementation of two base functions in Haskell, $f(x) := x^2$ and $g(x) := x + 1.$
These are pretty straightforward:
\inputminted[firstline=7,
             lastline=11,
             fontsize=\footnotesize, tabsize=4]{haskell}{ps1.hs}
In order to implement the and-then composition, we introduce a helper function, {\tt andThen} for $\fatsemi$:
\inputminted[firstline=13,
             lastline=14,
             fontsize=\footnotesize, tabsize=4]{haskell}{ps1.hs}
Now we can implement both $h = f \circ g$ and $i = f \fatsemi g$:
\inputminted[firstline=16,
             lastline=20,
             fontsize=\footnotesize, tabsize=4]{haskell}{ps1.hs}
and the answers are as follows: $h(2) = 9,\ i(2) = 5.$
%% ps1_q2.tex
\section{Two small categories}

Objects: 1, 2

Morphisms: 

1 -> 1: id1
2 -> 2: id2
1 -> 2: f
2 -> 1: (empty set)

Composition:
  id1 . id1 = id1
  id2 . id2 = id2
  f . id1 = f
  id2 . f = f

Right unit: f . id1 = f
Left unit: id2 . f = f

Associativity: as we only compose with identity, the associativity is trivial,

   id2 . f . id1 (the only 3 morphisms) = (id2 . f) . id1 = id2 . (f . id1) = f

%% ps1_q3.tex
\section{ Is it an isomorphism? }

Yes. Since $g \circ f : c \rightarrow c$ and $f \circ g : d \rightarrow d$ both must be valid morphisms
in the category, and the only morphisms $c \rightarrow c$ and $d \rightarrow d$ are the identities,
then $g \circ f = id_c$ and $f \circ g = id_d,$ q.e.d.

%% ps1_q4.tex
\section{ Almost categories. }

\lipsum[1]
%% 5
\section{ Monoids. }

\lipsum[1]


%% 6
\section{ Preorders }

\lipsum
%% 7
\section{ Church Booleans. }

\begin{align*} 
 True &= λx.(λy.x)     \\
False &= λx.(λy.y)     \\
  AND &= λp.(λq.(pq)p) \\
   OR &= λp.(λq.(pp)q) 
\end{align*}


\begin{align*} 
True AND False &= AND True False = λp.(λq.(pq)p) True False = \\
               &= (λq.(True q) True) False = (True False) True = \\ 
               &= (λx.(λy.x) False) True = λy.False True = False = λx.(λy.y) \\
\\
False OR True  &= OR False True = λp.(λq.(pp)q) False True  \\
               &= λq.(False False)q True = (False False) True = \\
               &= (λx.(λy.y) False) True = λy.y True = True = λx.(λy.x)
\end{align*} 


%% 8
\section{ Y Combinator. }

% https://mvanier.livejournal.com/2897.html

% See also http://r6.ca/blog/20060919T084800Z.html

$$
Y = λf.(λx.f(xx))(λx.f(xx))
$$

\begin{align*} 
Y\ g &= (λx.g(xx))(λx.g(xx)) \\
     &= g ((λx.g(xx)) (λx.g(xx))) \\
     &= g (Y\ g) \\
     &= g (g (g (g (g \ldots))))
\end{align*} 
and it never ends if we use it like that. Still, if we take (using Haskell notation)
\begin{minted}{haskell}
g h x | x == 1    = 1
      | otherwise = x * (h $ x - 1)
\end{minted}
then for example 
\begin{align*} 
Y\ g\ 4 &= g\ (Y\ g)\ 4                \\
        &= 4 * ((Y\ g)\ 3)             \\
        &= 4 * (g\ (Y\ g)\ 3)          \\
        &= 4 * 3 * ((Y\ g)\ 2)         \\
        &= 4 * 3 * (g\ (Y\ g)\ 2)      \\
        &= 4 * 3 * 2 * ((Y\ g)\ 1)     \\
        &= 4 * 3 * 2 * (g\ (Y\ g)\ 1)  \\
        &= 4 * 3 * 2 * 1               \\
        &= 24,\text{ also known as } 4!
\end{align*} 

Here, we have a definition of a recursive function (factorial) in a language that does
not directly allow recursion. Of course, that requires lazy evaluation ---
and the definitions for naturals and multiplication.

%% 9
\section{ Defining a toy category in Haskell. }

Since we only have two oblects and three arrows, we can simply encode those as ADTs. 
To encode objects, we need to choose between a standard \mintinline{haskell}{Either () ()} and a custom data type;
the latter seems to be more readable:

\inputminted[firstline=28,
             lastline=56,
             fontsize=\footnotesize, tabsize=4]{haskell}{ps1.hs}

It is easy to see that the code closely follows the answer to question 2.
%% ps1_q10.tex
\section{ Grade the pset. }

\lipsum


\section{ References }

Vadim Radionov vadim.radionov@gmail.com private conversations

Terence Tao Analysis I

% GHC base 4.12.0.0 sources, https://hackage.haskell.org/package/base-4.12.0.0/docs/src/GHC.Base.html#%2B%2B


\bibliography{Analysis_I}
\bibliographystyle{plainnat}

\end{document}
