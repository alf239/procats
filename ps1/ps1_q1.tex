%% ps1_q1.tex
\section{ Functions in mathematics and Haskell. } 
We're tasked with implementation of two base functions in Haskell, $f(x) := x^2$ and $g(x) := x + 1.$
These are pretty straightforward:
\inputminted[firstline=7,
             lastline=11,
             fontsize=\footnotesize, tabsize=4]{haskell}{ps1.hs}
In order to implement the and-then composition, we introduce a helper function, {\tt andThen} for $\fatsemi$:
\inputminted[firstline=13,
             lastline=14,
             fontsize=\footnotesize, tabsize=4]{haskell}{ps1.hs}
Now we can implement both $h = f \circ g$ and $i = f \fatsemi g$:
\inputminted[firstline=16,
             lastline=20,
             fontsize=\footnotesize, tabsize=4]{haskell}{ps1.hs}
and the answers are as follows: $h(2) = 9,\ i(2) = 5.$